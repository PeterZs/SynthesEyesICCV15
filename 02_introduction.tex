%!TEX root = 00_main.tex

\section{Introduction}

% \commentA{would be nice to show the process figure as a teaser above abstract as in the disney paper}
% Looks like the template doesn't allow us to do this easily unfortunately :(

Machine learning methods that leverage large amounts of training data currently perform best for many problems in computer vision, such as object detection, scene recognition, or gaze estimation~\cite{zhou2014learning,girshick2014rich,zhang15_cvpr}.
However, capturing or collecting large-scale training data can be extremely time-consuming, especially for new areas of research without pre-existing datasets.
In addition, supervised learning methods require accurate ground truth annotation for each image.
This annotation process can be expensive and tedious, and there is no guarantee that human-provided labels will be correct.
Ground truth annotation is particularly challenging and error-prone for learning tasks that require highly fine-grained and accurate labels, such as tracking facial landmarks for facial expression analysis and gaze estimation~\todo{REF}, or body joints for pose estimation and activity recognition~\todo{REF}.

\begin{figure}
    \includegraphics[width=\columnwidth]{teaser}
    \caption{At the core of our method is a dynamic eye-region model that allows us to render large numbers of photorealistic eye images as training data for eye region registration and gaze estimation.}
    \label{fig:teaser}
\end{figure}

% \commentE{The further back I look, rendering RGB training images does go back into the past for pose-estimation at least, though none of it was ``photorealistic''}
To address these problems, researchers have employed \emph{learning-by-synthesis} techniques to generate large amounts training data with computer graphics.
The advantages of this approach are that both data collection and annotation require little human labour and image synthesis can be geared to specific application scenarios.
Perhaps the best known example is by \citet{shotton2013real}, who synthesized a large and varied depth-image dataset for training a real-time pose-estimator.

Eyes and their movements are important for a range of applications including gaze-based human-computer interaction, visual behaviour monitoring, and deception analysis. \todo{cites} Recent work by \citet{sugano2014learning} demonstrated the benefits of learning-by-synthesis for gaze estimation, but employed only fundemental computer graphics techniques for synthesis.
% One part of the human body largely neglected in the context of image synthesis so far is the face and particularly the eyes.
% This is despite the fact that the eyes are important for a range of applications in different areas including gaze-based human-computer interaction, \todo{add more and refs}.
% probably don't want to stress the number of possible deformations, we only model natural eyelid movement, not eye-scrunching etc.
% maybe split up the sentence below
The eye-region is particularly difficult to model accurately given the dynamic shape changes it undergoes with facial motion and eyeball rotation, the complex material structure of the eyeball itself, and the significant variation in facial shape and texture across different people.

\todo{final sentence on limitations of existing models in this area}

\commentA{we also need to motivate why high quality renderings are needed, i.e. why it's worthwhile to put so much effort into the model. And later we can hopefully also show that it pays off to do so...}

% Andreas: we need some transition to eyeballs here, e.g. eyeball rendering is particularly challenging and interesting because of the many muscles involved, the large number of appearance details around and in the eye etc.
% essentially motivate that this hasn't been done before and is a very interesting area of research

%Synthesising training data is not novel in itself -- previous work has ... Our novel approach 

In this paper we present a novel method for photorealistic rendering of full face and eye images at a large scale.
Our method \todo{briefly summarise key characteristics of the method}
We describe how we prepare a collection of dynamic eye-region models, and then our approach for generating large amounts of photorealistic training data.
We then present and evaluate two separate systems trained on \dataset: a novel eye-region specific deformable model and an appearance-based gaze estimator.
These systems are case studies that show how we leverage the degrees of control made available by rendering our training data to easily and quickly generate high quality training datasets.
We show that by rendering all our training data, we achieve state of the art performance...

The specific contributions of this work are threefold. First, \todo{...}

