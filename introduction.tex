%!TEX root = syntheyes15.tex

\section{Introduction}

Machine learning approaches that leverage large amounts of image data are currently the best solutions to many problems in computer vision \todo{cite}. However, capturing or collecting images can be extremely time consuming, especially for new areas of research without pre-existing datasets. Supervised learning approaches then require that the images are labelled. This annotation process can be expensive and tedious, and there is no guarantee the labels will be correct.

Instead we show that by rendering all our training data, we achieve state of the art performance...

Synthesising training data is not novel in itself -- previous work has ... Our novel approach 

In this paper we describe how we prepare a collection of dynamic eye-region models, and then our approach for generating large amounts of photorealistic training data. We then present and evaluate two separate systems trained on \dataset: a novel eye-region specific deformable model and an appearance-based gaze estimator.
%
These systems are case studies that show how we leverage the degrees of control made available by rendering our training data to easily and quickly generate high quality training datasets.

\begin{figure}
    \includegraphics[width=\columnwidth]{main_img}
    \caption{We render images.}
\end{figure}

