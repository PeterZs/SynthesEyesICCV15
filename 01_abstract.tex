%!TEX root = 00_main.tex
Images of the eye are key in several computer vision problems, such as facial feature localization and gaze estimation.
%, iris-biometrics, and deception analysis.
Recent large-scale supervised methods for these problems require time-consuming data collection and manual annotation, which can be unreliable.
%, which is error-prone and slows down progress in these areas.
We propose synthesizing perfectly labelled photo-realistic training data in a fraction of the time.
We used computer graphics techniques to build a collection of dynamic eye-region models from head scan geometry.
%
These were randomly posed to synthesize close-up eye images for a wide range of head poses, gaze directions, and illumination conditions.
%
% We validated the usefulness of the dataset on two scenarios: an eye landmark detector and an appearance-based gaze estimator.
% The model is able to simulate the large variability of real eyes, including pupil dilation, eyelid motion and corresponding changes in its shape, as well as iris colour variations.
%dynamic changes in shape
We used our model's controllability to verify the importance of realistic illumination and shape variations in eye-region training data.
%
Finally we demonstrate the benefits of our synthesized training data (\dataset) by out-performing state-of-the-art methods for eye-shape registration in the wild, and achieving competitive performance on appearance-based gaze estimation.
% needs reword.
