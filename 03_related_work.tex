%!TEX root = syntheyes15.tex

\section{Related Work}

Our work is related to previous works on 1) learning using synthetic data as well as 2) computer generated eyes.

\subsection{Learning Using Synthetic Data}

\cite{stratou2011effect} -- relit 3d face scans to study the effect of illumination on automatic expression recognition.

\cite{sugano2014learning} -- learning by synthesis for 3D gaze estimation

\cite{lu2012head}

\cite{fanelli2011real} -- train head pose estimator on only synthetic depth data.

\subsection{Computer Generated Eyes}

\commentA{the ruhland2014look reference contains lots of additional relevant references}
\cite{ruhland2014look}

\commentA{could also cover geometric eye models, e.g. as used in eye tracking}
%\cite{bohme2008software}

The eyeballs are complex organs comprised of multiple layers of tissue, each with different reflectance properties and levels of transparency.
Fortunately, given that rendering realistic eyes is important for many areas in computer graphics (CG), there is already a large body of previous work on modelling and rendering eyes (see~\cite{ruhland2014look} for a recent survey).

\cite{feng1998variance} -- very early work on synthesising eye images

\cite{berard2014highquality} -- state of the art Disney approach

% \cite{alabort2014statistically} -- trained a detailed deformable eye region model on in-the-wild images.

\cite{priamikov14_openeyesim} -- 3D biomechanical model of the human extra-ocular eye muscles

\cite{swirski2014rendering} -- uses rendered videos of eyes to evaluate eye tracking algorithms.

% \subsection{Gaze estimation}

% \cite{xiong2014gaze} -- regression with features of 3d pupil centers and eye-contours (the eyelids) for gaze estimation.  Use multiple cameras and IR lights.

% \cite{zhang15_cvpr} -- multimodal CNNs for appearance-based gaze estimation in the wild

% eyediap
