%!TEX root = 00_main.tex

\section{Related Work}

Our work is related to previous works on 1) learning using synthetic data as well as 2) computer generated eyes.

\subsection{Learning Using Synthetic Data}

While learning-based approaches have been recognised as promising solutions for various computer vision tasks, the gap between training and test data still causes practical issues.
It is still not trivial for learning algorithms to handle unknown test data, and their performances often depend on how well the test data distribution is covered by the training data.
One straightforward solution is to use training data that covers the whole variation of potential test data; however, it is not always possible to obtain real training data in this manner.
The use of purely synthetic training data has been considered as a promising solution to address this issue.
It has been shown that synthetic training data is beneficial to various tasks including body pose estimation~\cite{shakhnarovich2003fast,okada2008relevant,shotton2013real}, head pose estimation~\cite{fanelli2011real} and object detection/recognition~\cite{yu2010improving,liebelt2010multiview,jaderberg2014synthetic}.
These works used 3D models, from a single 3D CAD object to deformable human body shape, to synthesise huge amount of training images for the target task.
% \commentY{CAD-based multi-view object recognition has a lot of works, though I don't think we need to cite them all}

As discussed by \citet{kaneva2011evaluation} \commentY{In some sense this is an important reference for our story}, one of the most important factors to consider is the realism of synthesised training images.
If the object of interest is highly complex, like the human eye, it is not clear whether we can rely on overly-simplistic object models.
Most similar to this work, \citet{sugano2014learning} used 3D reconstructions of eye regions to synthesise multi-view training data for appearance-based gaze estimation.
However, \citet{zhang15_cvpr} revealed that estimation accuracy significantly drops when the test data is obtained from a completely new environment.
Similarly to facial expression recognition~\cite{stratou2011effect}, illumination effects are a critical factor for computer vision, and the change of viewpoints is not enough to cover the test data variability.
\commentY{Is the following line correct? I thought they are just evaluating robustness of their approach against these factors}
\citet{zface} varied ambient and directional lighting conditions during synthesis to achieve illumination invariance.
\commentY{this line should be adjusted according to the result}
In contrast, our model allows...

\commentY{reality and controllability should be both important, while currently the story focuses on the reality side. but this point is rather related to the next section (Computer Generated Eyes) maybe?}
Another limitation of Sugano et al.'s dataset~\cite{sugano2014learning} is that they do not provide a parametric model.
Their data is essentially a set of rigid and low-resolution 3D models of eye regions with their ground-truth gaze directions, and hence cannot be easily applied to different tasks.
The scope of a realistic eye model for learning-based approach is not limited to the appearance-based gaze estimation task, but the model can be also applied to, e.g., eye shape registration task.
Since our model is fully synthetic, it can be also used to synthesise close-up eye images with ground-truth eye landmark positions.
This enables us to address the eye shape registration task via the learning-by-synthesis approach for the first time.

%\cite{okada2008relevant}, \cite{shakhnarovich2003fast} -- both rendered RGB training images with poser for pose recognition
%\cite{fanelli2011real} -- train head pose estimator on only synthetic depth data.
%\cite{stratou2011effect} -- relit 3d face scans to study the effect of illumination on automatic expression recognition.
%\cite{sugano2014learning} -- learning by synthesis for 3D gaze estimation

\commentY{(As one of the authors) I think this approach is in a little different context and we don't necessarily need to mention here:} \cite{lu2012head}

\subsection{Computer Generated Eyes}

\commentA{first paragraph: other body parts, then
second paragraph: In contrast, the problem of modelling the human eye has received considerably less attention.
}
\commentA{the ruhland2014look reference contains lots of additional relevant references}
\cite{ruhland2014look}

\commentA{could also cover geometric eye models, e.g. as used in eye tracking}
%\cite{bohme2008software}

The eyeballs are complex organs comprised of multiple layers of tissue, each with different reflectance properties and levels of transparency.
Fortunately, given that rendering realistic eyes is important for many areas in computer graphics (CG), there is already a large body of previous work on modelling and rendering eyes (see~\cite{ruhland2014look} for a recent survey).

\cite{feng1998variance} -- very early work on synthesising eye images

\cite{berard2014highquality} -- state of the art Disney approach

% \cite{alabort2014statistically} -- trained a detailed deformable eye region model on in-the-wild images.

\cite{priamikov14_openeyesim} -- 3D biomechanical model of the human extra-ocular eye muscles

% As we spend so much time looking at eyes, mistakes in their appearance can cause a CG face to appear unfamiliar.

CG eyes are important for the video-game industry, who want to model them across different characters and with potentially dramatic appearance (e.g. crying). \citet{ActiBlizEyes} developed techniques for real-time modelling of eye wetness, refraction, and ambient occlusion in a standard rasterization pipeline, as well as approximations for light-transfer properties of eyelids.
 
CG eyes have also been used to evalaute geometric gaze estimation algorithms, allowing individual parts of an eye-tracking system to be evaluated individually, rather than as an entire system \cite{bohme2008software,swirski2014rendering}. \citet{swirski2014rendering} used a rigged head model and reduced eyeball model to render ground truth images for pupil tracking algorithms.

% \subsection{Gaze estimation}

% \cite{xiong2014gaze} -- regression with features of 3d pupil centers and eye-contours (the eyelids) for gaze estimation.  Use multiple cameras and IR lights.
