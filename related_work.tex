%!TEX root = syntheyes15.tex

\section{Related work}

\subsection{Synthetic data}

\cite{swirski2014rendering} -- uses rendered videos of eyes to evaluate eye tracking algorithms.

\cite{stratou2011effect} -- relit 3d face scans to study the effect of illumination on automatic expression recognition.

\cite{sugano2014learning} -- learning by synthesis for 3D gaze estimation

\cite{fanelli2011real} -- train head pose estimator on only synthetic depth data.

\subsection{Deformable eye model}

The eyeballs are complex organs comprised of multiple layers of tissue, each with different reflectance properties and levels of transparency.
Fortunately, given that rendering realistic eyes is important for many areas in computer graphics, there is already a large body of previous work on modelling and rendering eyes \cite{ruhland2014look}.

\cite{alabort2014statistically} -- trained a detailed deformable eye region model on in-the-wild images.

\cite{priamikov14_openeyesim} -- 3D biomechanical model of the human extra-ocular eye muscles

\subsection{Gaze estimation}

\cite{xiong2014gaze} -- regression with features of 3d pupil centers and eye-contours (the eyelids) for gaze estimation.  Use multiple cameras and IR lights.

\cite{zhang15_cvpr} -- multimodal CNNs for appearance-based gaze estimation in the wild

% eyediap
